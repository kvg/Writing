\chapter{Methods}
\label{ch:methods}

\section{Lit review}
\subsection{Review of Kong et al., 2002}

Augustine Kong et al. discuss a new genetic map of recombination rates using genotyping information from $869$ individuals in 146 Icelandic families.  This is the first such map made after the sequencing of the human genome, and is thus able to leverage the new reference sequence in order to correctly order the genotyped markers.  It is a substantially higher-resolution map than provided by the former gold-standard, the Marshfield map.  The Marshfield map contained data on only $188$ meioses, whereas the Kong et al. map contained data on $1,257$.  The new map reveals marked differences in recombination rates between males and females (e.g. the recombination rate in female autosomes is a factor of $1.65$ higher than that observed in males) for reasons beyond sequence features.

